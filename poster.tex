% !TEX encoding = UTF-8 Unicode
% !TeX TS-program = lualatex
% !TeX spellcheck = en-US
% !BIB TS-program = bibtex
% -*- coding: UTF-8; -*-
% vim: set fenc=utf-8
%
\newcommand{\FirstLP}{Laurent Udo}
\newcommand{\LastLP}{Perrinet}
\newcommand{\AuthorLP}{\FirstLP \LastLP}
\newcommand{\EmailLP}{laurent.perrinet@univ-amu.fr}
\newcommand{\orcidLP}{0000-0002-9536-010X}
\newcommand{\Department}{Institut de Neurosciences de la Timone}% (UMR 7289)}%
\newcommand{\Affiliation}{Aix Marseille Univ, CNRS}%
\newcommand{\Street}{27 boulevard Jean Moulin}%
\newcommand{\PostCode}{13005}%
\newcommand{\City}{Marseille}%
\newcommand{\Country}{France}%
\newcommand{\WebsiteLP}{https://laurentperrinet.github.io}%
\newcommand{\Keywords}{time code, event-based computations, spiking neural networks, motion detection, efficient coding, logistic regression
}
\newcommand{\Funding}{
This research was funded by the European Union ERA-NET CHIST-ERA 2018 research and innovation program under grant agreement ANR-19-CHR3-0008-03 (``\href{APROVIS3D}{http://aprovis3d.eu/}''). 
LP received funding from the ANR project ANR-20-CE23-0021 (``\href{AgileNeuroBot}{https://laurentperrinet.github.io/grant/anr-anr/}'')
and from A*Midex grant number AMX-21-RID-025
(``\href{Polychronies}{https://laurentperrinet.github.io/grant/polychronies/}''). 
}
\newcommand{\Acknowledgments}{
The authors thank Salvatore Giancani, Hugo Ladret, Camille Besnainou, Jean-Nicolas Jérémie, Miles Keating, and Adrien Fois for useful discussions during the elaboration of this work. 
\Funding %
%For the purpose of open access, the author has applied a CC BY public copyright licence to any Author Accepted Manuscript version arising from this submission. This research was funded, in whole or in part, by [Organisation name, Grant #]. 
A CC-BY public copyright license has been applied by the authors to the present document and will be applied to all subsequent versions up to the Author Accepted Manuscript arising from this submission, in accordance with the grant’s open access conditions. 
}
\newcommand{\DataAvailability}{
This work is made reproducible. The code reproducing the manuscript and all figures is available on~\url{https://github.com/SpikeAI/2023_GrimaldiPerrinet_HeterogeneousDelaySNN}. It also contains supplementary figures and results. Find also the associated zotero group used to gather relevant literature on the subject at~\url{https://www.zotero.org/groups/4776796/fastmotiondetection}.
}
%
%% Gemini theme
% https://github.com/anishathalye/gemini

\documentclass[final]{beamer}

% ====================
% Packages
% ====================

\usepackage[T1]{fontenc}
\usepackage{lmodern}
\usepackage[size=custom,width=120,height=72,scale=1.0]{beamerposter}
\usetheme{gemini}
\usecolortheme{gemini}
\usepackage{graphicx}
\usepackage{booktabs}
\usepackage{tikz}
\usepackage{pgfplots}
\pgfplotsset{compat=1.14}
\usepackage{anyfontsize}
% \usepackage{natbib}
\usepackage{siunitx}%The siunitx package provides a  set  of  tools  for  authors  to  typeset  quantities  in  aconsistent  way.

\newcommand{\ms}{\si{\milli\second}}%


 \usepackage[
 %style=chem-acs,
 style=numeric,						% numeric style for reference list
 citestyle=numeric-comp,
 %style=alphabetic-verb,
 giveninits=false,
 maxbibnames=1,
 %firstinits=true,
 %style=apa,
 %maxcitenames=1,
 %maxnames=3,
 %minnames=1,
 %maxbibnames=99,
% %maxbibnames=99,
% dateabbrev=true,
% giveninits=true,
 %uniquename=init,
% url=false,
% doi=false,
% isbn=false,
% eprint=false,
% texencoding=utf8,
% bibencoding=utf8,
% autocite=superscript,
% backend=biber,
 %sorting=none,
 sorting=none,
 sortcites=false,
 %articletitle=false
 ]{biblatex}%

 \bibliography{poster}

% ====================
% Lengths
% ====================

% If you have N columns, choose \sepwidth and \colwidth such that
% (N+1)*\sepwidth + N*\colwidth = \paperwidth
\newlength{\sepwidth}
\newlength{\colwidth}
\setlength{\sepwidth}{0.025\paperwidth}
\setlength{\colwidth}{0.3\paperwidth}

\newcommand{\separatorcolumn}{\begin{column}{\sepwidth}\end{column}}

% ====================
% Title
% ====================

\title{Accurate Detection of Spiking Motifs in Neurobiological Data by Learning Heterogeneous Delays
% of a Spiking Neural Network
}

\author{\FirstLP \LastLP}

\institute[shortinst]{\inst{1} Some Institute \samelineand \inst{2} Another Institute}

% ====================
% Footer (optional)
% ====================

\footercontent{
  \href{https://laurentperrinet.github.io/publication/perrinet-24-fens/}{https://laurentperrinet.github.io/publication/perrinet-24-fens/} \hfill
  ABC Conference 2025, New York --- XYZ-1234 \hfill
  \href{mailto:alyssa.p.hacker@example.com}{alyssa.p.hacker@example.com}}
% (can be left out to remove footer)

% ====================
% Logo (optional)
% ====================

% use this to include logos on the left and/or right side of the header:
% \logoright{\includegraphics[height=7cm]{logo1.pdf}}
% \logoleft{\includegraphics[height=7cm]{logo2.pdf}}

% ====================
% Body
% ====================

\begin{document}

\begin{frame}[t]
\begin{columns}[t]
\separatorcolumn

\begin{column}{\colwidth}

  \begin{block}{A block title}

\begin{figure}[H]
%\centering
\includegraphics[width=.75\textwidth]{figures/visual-latency.pdf}
% thorpe
% \includegraphics[width=.7\textwidth]{figures/visual-latency-estimate.jpg}%-estimate.jpg}
\caption{
  Latency of the different processing steps along the human visual pathway.Though the visual system is highly inter-connected, one can follow the sequence of activations whenever an image (here a yellow star) is flashed in front of the eyes. Different areas are schematically represented by ellipses, and arrows denote the fastest feed-forward activation, ordered with respect to their activation latency in $\ms$. In order, the retina is first activated (20--40~$\ms$), then the thalamus and the primary visual cortex (V1, 60--90~$\ms$). This visual information  projects to the temporal lobe to reach the infero-temporal area (IT, 150~$\ms$) for object recognition. It then reaches the prefrontal cortex (PFC, $180~\ms$), which modulates decision making to evoke the motor cortex (MC, $220~\ms$) which may mediate an action. This is eventually relayed through the spinal cord to trigger finger muscles, with latencies of about 280--400~$\ms$.}\label{fig:thorpe}
\end{figure}


    Some block contents, followed by a diagram, followed by a dummy paragraph. %~\textcite{Grimaldi22polychronies}

%    \begin{figure}
%      \centering
%      \begin{tikzpicture}[scale=6]
%        \draw[step=0.25cm,color=gray] (-1,-1) grid (1,1);
%        \draw (1,0) -- (0.2,0.2) -- (0,1) -- (-0.2,0.2) -- (-1,0)
%          -- (-0.2,-0.2) -- (0,-1) -- (0.2,-0.2) -- cycle;
%      \end{tikzpicture}
%      \caption{A figure caption.}
%    \end{figure}

    \begin{figure}[H]%[t!]
      %  \centering ~\textcite{izhikevich_polychronization_2006}
        \includegraphics[width=0.980\linewidth]{figures/izhikevich.pdf}%png}% https://www.overleaf.com/5625872443qpcwrkssgbsf
          \caption{\textbf{\hl{Core mechanism of polychrony detection}.} {(\textbf{Left})}~In this example, three presynaptic neurons denoted \textit{b}, \textit{c} and, \textit{d} are fully connected to two post-synaptic neurons \textit{a} and \textit{e}, with different delays of respectively $1$, $5$, and $9~\ms$ for \textit{a} and  $8$, $5$, and $1~\ms$ for \textit{e}. {(\textbf{Middle})}~If three synchronous pulses are emitted from presynaptic neurons, this will generate post-synaptic potentials that will reach \textit{a} and \textit{e} asynchronously because of the heterogeneous delays, and they may not be sufficient to reach the membrane threshold in either of the post-synaptic neurons; therefore, no spike will be emitted, as this is not sufficient to reach the membrane threshold of the post synaptic neuron, so no output spike is emitted.
          %at these different delays, and these may not be sufficient to generate a spike in either neuron.
          {(\textbf{Right})}~If the pulses are emitted from presynaptic neurons such that, taking into account the delays, they reach the post-synaptic neuron \textit{a} at the same time (here, at $t=10~\ms$),  the post-synaptic potentials evoked by the three pre-synaptic neurons sum up, causing the voltage threshold to be crossed and thus to the emission of an output spike (red color), while none is emitted from post-synaptic neuron \textit{e}.
           }
        \label{fig:izhikevich}
      \end{figure}
      %

    Lorem ipsum dolor sit amet, consectetur adipiscing elit. Morbi ultricies
    eget libero ac ullamcorper. Integer et euismod ante. Aenean vestibulum
    lobortis augue, ut lobortis turpis rhoncus sed. Proin feugiat nibh a
    lacinia dignissim. Proin scelerisque, risus eget tempor fermentum, ex
    turpis condimentum urna, quis malesuada sapien arcu eu purus.

  \end{block}

%  \begin{block}{A block containing a list}
%
%    Nam vulputate nunc felis, non condimentum lacus porta ultrices. Nullam sed
%    sagittis metus. Etiam consectetur gravida urna quis suscipit.
%
%    \begin{itemize}
%      \item \textbf{Mauris tempor} risus nulla, sed ornare
%      \item \textbf{Libero tincidunt} a duis congue vitae
%      \item \textbf{Dui ac pretium} morbi justo neque, ullamcorper
%    \end{itemize}
%
%    Eget augue porta, bibendum venenatis tortor.
%
%  \end{block}
%
%  \begin{alertblock}{A highlighted block}
%
%    This block catches your eye, so \textbf{important stuff} should probably go
%    here.
%
%    Curabitur eu libero vehicula, cursus est fringilla, luctus est. Morbi
%    consectetur mauris quam, at finibus elit auctor ac. Aliquam erat volutpat.
%    Aenean at nisl ut ex ullamcorper eleifend et eu augue. Aenean quis velit
%    tristique odio convallis ultrices a ac odio.
%
%    \begin{itemize}
%      \item \textbf{Fusce dapibus tellus} vel tellus semper finibus. In
%        consequat, nibh sed mattis luctus, augue diam fermentum lectus.
%      \item \textbf{In euismod erat metus} non ex. Vestibulum luctus augue in
%        mi condimentum, at sollicitudin lorem viverra.
%      \item \textbf{Suspendisse vulputate} mauris vel placerat consectetur.
%        Mauris semper, purus ac hendrerit molestie, elit mi dignissim odio, in
%        suscipit felis sapien vel ex.
%    \end{itemize}
%
%    Aenean tincidunt risus eros, at gravida lorem sagittis vel. Vestibulum ante
%    ipsum primis in faucibus orci luctus et ultrices posuere cubilia Curae.
%
%  \end{alertblock}

\end{column}

\separatorcolumn

\begin{column}{\colwidth}

  \begin{block}{A block containing an enumerated list}

    Vivamus congue volutpat elit non semper. Praesent molestie nec erat ac
    interdum. In quis suscipit erat. \textbf{Phasellus mauris felis, molestie
    ac pharetra quis}, tempus nec ante. Donec finibus ante vel purus mollis
    fermentum. Sed felis mi, pharetra eget nibh a, feugiat eleifend dolor. Nam
    mollis condimentum purus quis sodales. Nullam eu felis eu nulla eleifend
    bibendum nec eu lorem. Vivamus felis velit, volutpat ut facilisis ac,
    commodo in metus.

    \begin{enumerate}
      \item \textbf{Morbi mauris purus}, egestas at vehicula et, convallis
        accumsan orci. Orci varius natoque penatibus et magnis dis parturient
        montes, nascetur ridiculus mus.
      \item \textbf{Cras vehicula blandit urna ut maximus}. Aliquam blandit nec
        massa ac sollicitudin. Curabitur cursus, metus nec imperdiet bibendum,
        velit lectus faucibus dolor, quis gravida metus mauris gravida turpis.
      \item \textbf{Vestibulum et massa diam}. Phasellus fermentum augue non
        nulla accumsan, non rhoncus lectus condimentum.
    \end{enumerate}

  \end{block}

  \begin{block}{Fusce aliquam magna velit}

    Et rutrum ex euismod vel. Pellentesque ultricies, velit in fermentum
    vestibulum, lectus nisi pretium nibh, sit amet aliquam lectus augue vel
    velit. Suspendisse rhoncus massa porttitor augue feugiat molestie. Sed
    molestie ut orci nec malesuada. Sed ultricies feugiat est fringilla
    posuere.

    \begin{figure}
      \centering
      \begin{tikzpicture}
        \begin{axis}[
            scale only axis,
            no markers,
            domain=0:2*pi,
            samples=100,
            axis lines=center,
            axis line style={-},
            ticks=none]
          \addplot[red] {sin(deg(x))};
          \addplot[blue] {cos(deg(x))};
        \end{axis}
      \end{tikzpicture}
      \caption{Another figure caption.}
    \end{figure}

  \end{block}

  \begin{block}{Nam cursus consequat egestas}

    Nulla eget sem quam. Ut aliquam volutpat nisi vestibulum convallis. Nunc a
    lectus et eros facilisis hendrerit eu non urna. Interdum et malesuada fames
    ac ante \textit{ipsum primis} in faucibus. Etiam sit amet velit eget sem
    euismod tristique. Praesent enim erat, porta vel mattis sed, pharetra sed
    ipsum. Morbi commodo condimentum massa, \textit{tempus venenatis} massa
    hendrerit quis. Maecenas sed porta est. Praesent mollis interdum lectus,
    sit amet sollicitudin risus tincidunt non.

    Etiam sit amet tempus lorem, aliquet condimentum velit. Donec et nibh
    consequat, sagittis ex eget, dictum orci. Etiam quis semper ante. Ut eu
    mauris purus. Proin nec consectetur ligula. Mauris pretium molestie
    ullamcorper. Integer nisi neque, aliquet et odio non, sagittis porta justo.

    \begin{itemize}
      \item \textbf{Sed consequat} id ante vel efficitur. Praesent congue massa
        sed est scelerisque, elementum mollis augue iaculis.
        \begin{itemize}
          \item In sed est finibus, vulputate
            nunc gravida, pulvinar lorem. In maximus nunc dolor, sed auctor eros
            porttitor quis.
          \item Fusce ornare dignissim nisi. Nam sit amet risus vel lacus
            tempor tincidunt eu a arcu.
          \item Donec rhoncus vestibulum erat, quis aliquam leo
            gravida egestas.
        \end{itemize}
      \item \textbf{Sed luctus, elit sit amet} dictum maximus, diam dolor
        faucibus purus, sed lobortis justo erat id turpis.
      \item \textbf{Pellentesque facilisis dolor in leo} bibendum congue.
        Maecenas congue finibus justo, vitae eleifend urna facilisis at.
    \end{itemize}

  \end{block}

\end{column}

\separatorcolumn

\begin{column}{\colwidth}

  \begin{exampleblock}{A highlighted block containing some math}

    A different kind of highlighted block.

    $$
    \int_{-\infty}^{\infty} e^{-x^2}\,dx = \sqrt{\pi}
    $$

    Interdum et malesuada fames $\{1, 4, 9, \ldots\}$ ac ante ipsum primis in
    faucibus. Cras eleifend dolor eu nulla suscipit suscipit. Sed lobortis non
    felis id vulputate.

    \heading{A heading inside a block}

    Praesent consectetur mi $x^2 + y^2$ metus, nec vestibulum justo viverra
    nec. Proin eget nulla pretium, egestas magna aliquam, mollis neque. Vivamus
    dictum $\mathbf{u}^\intercal\mathbf{v}$ sagittis odio, vel porta erat
    congue sed. Maecenas ut dolor quis arcu auctor porttitor.

    \heading{Another heading inside a block}

    Sed augue erat, scelerisque a purus ultricies, placerat porttitor neque.
    Donec $P(y \mid x)$ fermentum consectetur $\nabla_x P(y \mid x)$ sapien
    sagittis egestas. Duis eget leo euismod nunc viverra imperdiet nec id
    justo.

  \end{exampleblock}

  \begin{block}{Nullam vel erat at velit convallis laoreet}

    Class aptent taciti sociosqu ad litora torquent per conubia nostra, per
    inceptos himenaeos. Phasellus libero enim, gravida sed erat sit amet,
    scelerisque congue diam. Fusce dapibus dui ut augue pulvinar iaculis.

    \begin{table}
      \centering
      \begin{tabular}{l r r c}
        \toprule
        \textbf{First column} & \textbf{Second column} & \textbf{Third column} & \textbf{Fourth} \\
        \midrule
        Foo & 13.37 & 384,394 & $\alpha$ \\
        Bar & 2.17 & 1,392 & $\beta$ \\
        Baz & 3.14 & 83,742 & $\delta$ \\
        Qux & 7.59 & 974 & $\gamma$ \\
        \bottomrule
      \end{tabular}
      \caption{A table caption.}
    \end{table}

    Donec quis posuere ligula. Nunc feugiat elit a mi malesuada consequat. Sed
    imperdiet augue ac nibh aliquet tristique. Aenean eu tortor vulputate,
    eleifend lorem in, dictum urna. Proin auctor ante in augue tincidunt
    tempor. Proin pellentesque vulputate odio, ac gravida nulla posuere
    efficitur. Aenean at velit vel dolor blandit molestie. Mauris laoreet
    commodo quam, non luctus nibh ullamcorper in. Class aptent taciti sociosqu
    ad litora torquent per conubia nostra, per inceptos himenaeos.

    Nulla varius finibus volutpat. Mauris molestie lorem tincidunt, iaculis
    libero at, gravida ante. Phasellus at felis eu neque suscipit suscipit.
    Integer ullamcorper, dui nec pretium ornare, urna dolor consequat libero,
    in feugiat elit lorem euismod lacus. Pellentesque sit amet dolor mollis,
    auctor urna non, tempus sem.

  \end{block}

  \begin{block}{References}
    % \nocite{*}
    \footnotesize{
    %\bibliographystyle{plain}\bibliography{poster}
    \printbibliography[heading=none]
 }

  \end{block}

\end{column}

\separatorcolumn
\end{columns}
\end{frame}

\end{document}
